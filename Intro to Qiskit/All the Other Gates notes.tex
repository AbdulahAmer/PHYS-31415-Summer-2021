## Other Gates! 

In lecture you have already covered the Hadamard Gate. Which looks like this (in the $ \{ |0\rangle , |1\rangle \}$ basis)

$$
H=
\frac{1}{\sqrt{2}}
\begin{bmatrix}
1&1 \\
1&-1
\end{bmatrix}$$

and does this 

$$ H|0\rangle = |+\rangle = \frac{1}{\sqrt{2}} (|0\rangle + |1\rangle) $$ 
   
$$ H|1\rangle = |-\rangle = \frac{1}{\sqrt{2}} (|0\rangle - |1\rangle) $$
   
It behaves like a beam splitter by 'splitting' the amplitude evenly between our two basis states! 

This begs the question, what else can we do with our Qubits! Quite alot actually, so let us take a look at the other Quantum Gates and how they behave on a qubit. 


### X, Y, and Z 

Some very popular players in the Quantum Computing game are the Pauli Matrices, here is what they are and what they do! 

The X gate, also known as the NOT gate, 'flips' our qubit from one basis state to the other. 

$$
X=
\begin{bmatrix}
0&1 \\
1&0
\end{bmatrix} $$

$$ X|0\rangle = |1\rangle $$

$$ X|1\rangle = |0\rangle  $$

The Y gate, flips and adds a phase of $i$ or $-i$ to our state. 

$$
Y=
\begin{bmatrix}
0&-i \\
i&0
\end{bmatrix}$$

$$ Y|0\rangle = i|1\rangle $$
   
$$Y|1\rangle = -i|0\rangle  $$

The Z gate does nothing to the $|0\rangle$ state and multiplys the amplitude by -1 for the $|1\rangle$ state.

$$
Z=
\begin{bmatrix}
1&0 \\
0&-1
\end{bmatrix}$$

$$ Z|0\rangle = |0\rangle $$
   
$$ Z|1\rangle = -|1\rangle  $$
   

Lets try these out in Qiskit! 
